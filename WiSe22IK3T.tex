\documentclass[11pt]{report}

\usepackage[aufgaben,mitschrift,namen,design]{kern}

% \input{../../../../Verwaltung/LaTeX/Header/Konstruktionsdateien/kern.tex}
% \input{../../../../Verwaltung/LaTeX/Header/Konstruktionsdateien/mitschrift.tex}
% \input{../../../../Verwaltung/LaTeX/Header/Konstruktionsdateien/aufgabenpaket.tex}
% \input{../../../../Verwaltung/LaTeX/Header/Konstruktionsdateien/namensregister.tex}
% \input{../../../../Verwaltung/LaTeX/Header/Konstruktionsdateien/designaddon.tex}

\usepackage{geometry}
\geometry{
    letterpaper, 
    twoside=true, 
    bindingoffset=1cm, 
    left=3.5cm, 
    right=3.5cm, 
    top=3.5cm, 
    bottom=3.5cm
}

\ihead{\textbf{Integrierter Kurs IV}\\\textit{Experimentalphysik II}\\\texttt{Skript}}
\chead{\textit{Tom Folgmann}}

\setlength{\parindent}{0pt}

\title{Integrierter Kurs III}
\author{Theorieteil\\Mitschrift von Tom Folgmann}

\begin{document}
	\maketitle
	
	\subfile{Notes/lec1.tex}
		Sehr geehrte Damen und Herren, dies ist eine extrem lange Zeile, die ihren Job einzig in der Darstellung der Umbruchmethode hat, welche TeXShop verwendet. Vielen Dank f\"ur Ihr Verst\"andnis.
	\subfile{Notes/lec2.tex}
	\subfile{Notes/lec3.tex}
	\subfile{Notes/lec4.tex}
	\subfile{Notes/lec5.tex}
	\subfile{Notes/lec6.tex}
	\subfile{Notes/lec7.tex}
	\subfile{Notes/lec8.tex}
	\subfile{Notes/lec9.tex}
	\subfile{Notes/lec10.tex}
	\subfile{Notes/lec11.tex}
	\subfile{Notes/lec12.tex}
	\subfile{Notes/lec13.tex}
	\subfile{Notes/lec14.tex}
	\subfile{Notes/lec15.tex}
	\subfile{Notes/lec16.tex}
	\subfile{Notes/lec17.tex}
	\subfile{Notes/lec18.tex}
	\subfile{Notes/lec19.tex}
	\subfile{Notes/lec20.tex}
	
	\subfile{Notes/lec22.tex}
	
	\subfile{Notes/loesungen.tex}
\end{document}