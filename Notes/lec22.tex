\documentclass[../WiSe22ANA3.tex]{subfiles}
\begin{document}
	\lesson{1}{di 07 feb 2023 08:15}{Poisson-Klammern} 
		\textbf{Letzte Vorlesung! 🥳}
		Wir betrachten als letztes Kapitel des Semesters die sogenannten \emph{Poisson-Klammern}.
		\section{Poisson-Klammern}
			Seien $f,g\in C^1(\R\times(\R^d)^2,\R)$ Funktionen der Zeit $t\in\R$ und den Auswertungen der Parameterfunktionen $p,q\in C^1(\R,\R^d)$. Dabei kann $q$ der \emph{generalisierte Ort} und $p$ der \emph{generalisierte Impuls} sein. Dann können wir die Summe über die Freiheitsgrade $d\in\N$ des betrachteten Systems $\Sigma$ formulieren, als welcher Summanden wir die  symmetrische Differentialdifferenz
			\begin{multline*}
				\text{PD}_k(f,g):=\Diff{f}{0_\R,(e_k,0_{\R^d})}{x}{}\cdot\Diff{g}{0_\R,(0_{\R^d},e_k)}{x}{}\\
				-\Diff{f}{0_\R,(0_{\R^d},e_k)}{x}{}\cdot\Diff{g}{0_\R,(e_k,0_{\R^d})}{x}{}
			\end{multline*}
			mit der manchmal auch verwendeten Kurzschreibweise $\pdv{f}{q_k}\cdot\pdv{g}{p_k}-\pdv{f}{p_k}\cdot\pdv{g}{q_k}$. 
			\begin{info}[Poisson-Klammer]
				Als \emph{Poisson-Klammer} definieren wir den \emph{bilinearen Differentialoperator} in $\text{Diff}^1(C^1(\R\times(\R^d)^2,\R))$ der Form 
				$$\nset{f,g}:=\sum_{k\in[d]}\text{PD}_k(f,g)$$
				des Systems $\Sigma$ der Freiheitsgrade $d\in\N$. 
			\end{info}
			\begin{Aufgabe}
				\nr Zeige die Eigenschaften des bilinearen Differentialoperators: Zeige $\nset{\cdot,\cdot}$ stellt für die Funktionstypen einen Differentialoperator dar. Zeige dessen Bilinearität. 
			\end{Aufgabe}
			In der Menge $\text{Diff}^k(C^k(V,W))$ sammeln wir alle Differentialoperatoren auf dem Funktionenraum der $k$-mal stetig differenzierbaren Funktionen von $V$ nach $W$. 
			\begin{Aufgabe}
				\nr Recherchiere die genauen Rahmenbedingungen der Definition. 
			\end{Aufgabe}
			Die Poissonklammern stellen ein prominentes Beispiel für die sogenannten \emph{Lie-Klammern}, welche definiert sind als spezielle \emph{innere Verknüpfung} $[\cdot,\cdot]:V\times V\to V,\;(x,y)\mapsto [x,y]$ auf einem $K$-Vektorraum $V$. Sie erfüllt drei Charakteristiken, die wir im folgenden kurz beleuchten wollen. 
			\begin{info}[Lie-Klammer]
				Eine innere Verknüpfung $[\cdot,\cdot]:V^2\to V$ nennen wir \emph{Lie-Klammer}, wenn sie die Bedingungen 
				\begin{clist}
					\item Bilinearität,
					\item Es gilt $[x,x]=0_K$ für alle $x\in V$, 
					\item Sie ist mit der Jacobi-Rotationsidentität vereinbar; für $x,y,z\in V$ gilt 
 					$$[x,[y,z]]=[z,[x,y]]=[y,[z,x]],$$
				\end{clist}
				erfüllt. Liegt eine solche interne Verknüpfung auf einem $K$-Vektorraum $V$ vor, so wird $(V,[\cdot,\cdot])$ als \emph{Lie-Algebra} bezeichnet. 
			\end{info}
			
		\section{Theoreme}
			Zum Abschluss wollen wir noch einen Blick auf allgemeine Theoreme auf dem Vektorraum der Parameter der Form $(\R^d)^2$ mit $d\in\N$ als Freiheitsgrade betrachten. 
			\subsection*{Liouvilles Theorem}	
				Für eine Teilmenge $\Lambda\subseteq (\R^d)^2$ gilt die \emph{Volumentransformationsinvarianz} der Form 
				$$\lint{\psi(x)}{x}{(\R^d)^2}=\lint{(f\circ\psi)(x)\cdot V(f'(x))}{x}{(\R^d)^2}.$$
				Der Beweis führt über die Definition des \emph{Volumenfaktors} $V(f'(x))$ der \emph{kanonischen} Transformationsfunktion $f\in C^1((\R^d)^2,(\R^d)^2)$ durch die Gramsche Matrix   $G(x)=\fdef{\scpr{x_i}{x_j}}{(i,j)\in[N]^2}$ mit $N:=\dim((\R^d)^2)=2d$.
				\begin{Aufgabe}
					\nr Zeige den Satz mithilfe maßtheoretischer Produktmaße $\lambda^k$. 
				\end{Aufgabe}
				
			\subsection*{Hamilton-Jacobi Formalismus}
				Auf der Suche nach einer möglichst vorteilhaften Perspektive im Parameterraum, also Basis, versucht man eine optimierende Transformation $f$ zu finden, sodaß 
				$$H(t,f(x))=0.$$
				Dies ist genau dann möglich, wenn die Auswertung $f(x)$ nur \emph{Erhaltungsgrößen} enthält, also 
				$$\Diff{H}{0,E_i}{t,f(x)}{}=0$$
				für $E:=\fdef{\mbbEins_{[d]}\cdot(e_i,0_{\R^d})+\mbbEins_{\nset{d+1,2d}}\cdot (0_{\R^d},e_i)}{i\in[2d]}$ gilt. Behaupte nun die Nullbedingung:
				$$H(t,x)+H(t,h)=0.$$
				Dabei wählen wir nun \emph{speziell} die Form $h:=(x_1,\stern{f}{2}{x})$. Damit gilt dann
				\begin{align*}
					&x_2=\Diff{H}{0,(0,1_{\R^d})}{t,h}{} &&\stern{f}{1}{x}=\Diff{H}{0,(1_{\R^d},0)}{t,h}{}
				\end{align*}
				und in eingesetzter Form 
				$$H(t,(x_1,\Diff{H}{0,(0,1_{\R^d})}{t,h}{})+\Diff{H}{1,0_{(\R^d)^2}}{t,h}{}=0.$$
				\begin{Aufgabe}
					\nr Leite die skizzierte Beziehung aus der Wirkungsvariation der Lagrange Funktion $L$ gegeben durch $\deldiff{L}{0,(h(t),h'(t))}{t,(q(t),q'(t))}{}=0$ her. 
					
					\nr Beobachte unter welchen Bedingungen die Existenz einer solchen Transformationsfunktion gegeben ist.
				\end{Aufgabe}
				Das bereits skizzierte Programm zur Problemlösung folgt der Struktur:
				\begin{clist}
					\item Schreibe die Hamilton Funktion $H$ auf. 
					\item Finde die Transformation $f$. 
					\item Schreibe für die Form $h:=(x_1,\stern{f}{2}{x})$ die zweite Hamilton-Funktionsauswertung $H(t,h)$ auf. 
					\item Notiere die Variablen $x_1,x_2$ durch Ableitungen von $H(t,h)$. 
					\item Prüfe die Nullbedingung bei Addition. 
				\end{clist}
				\begin{Aufgabe}
					\nr Verwende das Programm, um per Hammerschlag das Problem des harmonischen Oszillators zu lösen. 
				\end{Aufgabe}
				
		\section*{Reiserückblick}
			Die Reise des \textt{IK3} führte uns von $3N\in\N$ Freiheitsgrade mit $k\in\N$ Zwangsbedingungen zu den generalisierten Koordinaten, welche wir mithilfe der Lagrange-Multiplikatoren haben. 
			
			Mit den Symmetriargumenten versuchten wir uns das Leben leichter zu machen, woraufhin wir zum ersten Höhepunkt die Lagrangegleichung aufstellten; Ein kurzer Blick auf die Hamiltonfunktion erlaubte uns einen solch umfangreichen Werkzeugkasten zu bedienen, daß Vielteilchensysteme mathematisch greifbar wurden. Wir schauten im kleinen nach Rotationen und Translationen und fanden Transformationsmatrizen.
			
			Als uns auffiel, daß mit hohen Geschwindigkeitstransformationen etwas nicht stimmte, fanden wir die spezielle Relativitätstheorie als Ausweg mit relativem Ausblick von Systemen auf andere mithilfe des Lorentz-Fernrohres. 
			
			Am Ende fanden wir den Weg zurück zu den Transformationen und den Blick in den Parameterraum und die Charakterisierung der kanonischen Transformationen. 
			
			Als weisere Wanderer verlassen wir nun die Theorie bis zum \textt{IK4}. 
\end{document}