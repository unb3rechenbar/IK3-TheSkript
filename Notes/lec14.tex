\documentclass[../WiSe22ANA3.tex]{subfiles}
\begin{document}
	\lesson{1}{di 10 jan 2023 08:30}{Fortsetzung Oszillator} 
		Wir nehmen nun an, die Eigenfrequenzen $\omega_\alpha$ seien paarweise verschieden. Wir untersuchen, ob $\Maus{A}{i,\alpha}$ proportional zu $\Minor{i}{\Lambda-\omega_\alpha^2T}$ ist. Wir sparen uns hier allerdings den allgemeinen Beweis und betrachten einen Spezialfall. Sei $n=3$ und $\alpha=1=i$. Dann 
		$$\sum_{j\in[3]}\Maus{\Lambda}{i,j}-\omega_\alpha^2\Maus{T}{i,j}\cdot\Maus{A}{j,\alpha}=\Null{\R^3}.$$
		\begin{Aufgabe}
			\nr Zeige den allgemeinen Fall. 
			
			\nr Überlege dir die Minoren der angegebenen Matrix für den $\R^3$ Fall. 
		\end{Aufgabe}
		
		Erweitere die Determinante entlang der ersten Spalte mit $\Maus{\mcM}{i,\alpha}:=\Minor{i}{\Lambda-\omega_\alpha^2T}$, also
		$$\sum_{i\in[3]}\nbra{\Maus{\Lambda}{i,1}-\omega_1^2\Maus{T}{i,1}}\cdot\Maus{\mcM}{i,\alpha}=0.$$
		Wir finden ein $C_\alpha\in\C$ mit
		$$\Maus{A}{i,\alpha}=C_\alpha\cdot\Maus{\mcM}{i,\alpha},$$
		charakterisiert durch 
		$$C_\alpha\cdot\nbra{\sum_{i\in[3]}\nbra{\Maus{\Lambda}{i,1}-\omega_1^2\Maus{T}{i,1}}\cdot\Maus{\mcM}{i,\alpha}}=0.$$
		Damit ergibt sich für die Verschiebung
		$$\Maus{h}{i,\alpha}(t)=C_\alpha\cdot\Maus{\mcM}{i,\alpha}\cdot\exp(-\cmath\omega_\alpha t),$$
		sodaß eine allgemeine reelle Lösung der Realteil von 
		$$\Real{h_i(t)}=\Real{\sum_{\alpha\in[n]}C_\alpha\cdot\Maus{\mcM}{i,\alpha}\cdot\exp(-\cmath\omega_\alpha t)}$$
		ist. 
		\begin{Aufgabe}
			\nr Rechne $\Maus{A}{i,\alpha}=C_\alpha\cdot\Maus{\mcM}{i,\alpha}$ nach. Erweitere hierzu zunächst die Determinante $\det(\Lambda-\omega_\alpha^2 T)$ in der ersten Spalte. 
			
			\nr Übertrage $\Maus{A}{i,\alpha}=C_\alpha\cdot\Maus{\mcM}{i,\alpha}$ auf die am Anfang betrachtete Verschiebung $h$ der generalisierten Koordinate $q$ im Potential $V$. 
			
			\nr Zeige, daß die allgemeine reelle Lösung die Gleichung (???) löst. 
		\end{Aufgabe}
		Schreiben wir $\Theta_\alpha(t):=\Maus{\mcM}{i,\alpha}\cdot\exp(-\cmath\omega_\alpha t)$, dann können wir überlegen, ob die Bewegungsgleichungen zu $\Theta_\alpha$ entkoppelt sind.
		\begin{Aufgabe}
			\nr Untersuche die Kopplung. 
		\end{Aufgabe}
		Damit dann 
		\begin{multline*}
			T(t,(h(t),h'(t)))=\frac{1}{2}\Real{\stern{h'}{i}{t}}\cdot\Real{\stern{h'}{j}{t}}\\
			=\frac{1}{2}\Maus{T}{i,j}\cdot\Maus{\mcM}{i,\alpha}\cdot\stern{\Theta'}{\alpha}{t}\cdot\Maus{\mcM}{j,\beta}\cdot\stern{\Theta'}{\beta}{t}
		\end{multline*}
		und 
		$$V(t,h(t))=\frac{1}{2}\Real{h_i}\Real{h_j}.$$
		\begin{Aufgabe}
			\nr Schreibe das Potential $V$ aus. 
		\end{Aufgabe}
		Wir können nun für $\alpha,\beta\in\Mengenschriftdesign{Eigenmodi}$ schreiben
		$$\sum_{j\in[n]}\Maus{\Lambda}{i,j}\Maus{\mcM}{i,\alpha}=\omega_\alpha^2\sum_{j\in[n]}\Maus{T}{i,j}\Maus{A}{j,\alpha}.$$
		\begin{Aufgabe}
			\nr Verwende $\Maus{A}{j,\alpha}=C_\alpha\cdot\Maus{\mcM}{i,\alpha}$ um die Summengleichheit zu zeigen. 
		\end{Aufgabe}
		Die Differenz für $\alpha,\beta$ ergibt dann
		$$\nbra{\omega_\alpha^2-\omega_\beta^2}\cdot\sum_{(i,j)\in[n]^2}\Maus{T}{i,j}\Maus{\mcM}{i,\alpha}\Maus{\mcM}{i,\beta}=0_\R.$$
		Wenn $\alpha\neq 0$ und $\beta\neq 0$ und $\alpha\neq\beta$ folgt $\sum_{(i,j)\in[n]^2}\Maus{T}{i,j}\Maus{\mcM}{i,\alpha}\Maus{\mcM}{i,\beta}=0$. Sind jedoch $\alpha=\beta$, dann kann die Summe Element aus $\R$ sein. 
		
		Irgendwie... kann man dann vereinfachen zu 
		\begin{align*}
			&T=\frac{1}{2}\nbra{\Theta_\alpha'(t)}^2 &&\frac{1}{2}\sum_{\alpha\in[n]}\omega_\alpha^2\Theta_\alpha(t)^2.
		\end{align*}
		\begin{Aufgabe}
			\nr Führe die Vereinfachung durch. 
		\end{Aufgabe}
		Damit kann die Lagrange-Gleichung endlich beschrieben werden durch
		$$L(t,(q(t),q'(t)))=\frac{1}{2}\sum_{\alpha\in[n]}\nbra{\Theta_\alpha'(t)^2-\omega_\alpha^2\Theta_\alpha(t)^2}$$
		mit der aus der \Euler-\Lagrange Gleichung resultierenden DGL
		$$\Theta_\beta''(t)+\omega_\beta^2\Theta_\beta(t)^2=0.$$
		\begin{Aufgabe}
			\nr Recherchiere \enquote{\emph{dispersion relation photons}} und betrachte die Graphen. 
			
			\nr Finde den Zusammenhang zur Festkörperphysik. Inwiefern ist der bearbeitete Zusammenhang z.B. im Metzler vorhanden? 
			
			\nr Recherchiere wie das Verfahren modifiziert wird, wenn zwei oder mehr Eigenfrequenzen entartet sind. 
		\end{Aufgabe}
		\begin{Beispiel}[Freies Teilchen]
			Betrachte ein freies Teilchen, dessen Potential $V(t,x(t))=\frac{1}{2}\nbra{k_1x_1^2+k_2x_2^2+k_3x_3^2}$ ist. Die kinetische Energie sei $T(t,(x(t),x'(t)))=\frac{1}{2}m\dabs{x'(t)}{2}^2$. Dann ist die Potentialenergiematrix $\Lambda$ diagonal, also $\Lambda=\fdef{\delta_(i,j)\cdot k_i}{(i,j)\in[3]^2}$. Die Matrix $T$ ist demensprechend $T=m\cdot I_3$. Es gilt also 
			$$m\stern{x''}{i}{t}+k_i\stern{x}{i}{t}=0$$
			für $i\in[3]$, wobei $\stern{x}{i}{t}=A_i\cdot\exp(-\cmath\omega t)$. Die Bedingung ist 
			$$\det(\Lambda-\omega^2 T)=0=\sum_{i\in[3]}\nbra{k_i-m\omega^2}$$
			und somit $\omega_i=\sqrt{k_i/m}$. 
		\end{Beispiel}
\end{document}