\documentclass[../WiSe22ANA3.tex]{subfiles}
\begin{document}
	\lesson{1}{di 29 nov 2022 08:45}{Hyperbolische Laufbahnen} 
		\section{Streuungsquerschnitt}
			
			
		\section{Beispiel Streuung geladener Teilchen}
			\begin{bsp}
				
				Es gilt die Exzentrizität
				\begin{align*}
					\mcE=...
				\end{align*}
			\end{bsp}
			Aus der vorangegangenen Analyse des \textsc{Kepler}-Problems folgt, daß allgemein gilt
			\begin{align*}
				\dabs{\vec r}{2}=\frac{p}{1+\mcE\cdot\cos(\beta-\beta_0)}\Longleftrightarrow\frac{p}{\dabs{r}{2}}=1+\mcE\cdot\cos(\beta-\beta_0). 
			\end{align*}
			
			
		\section{Streuparameter und Winkel}
			\begin{uups}
				Wir erhalten magisch 
				\begin{align*}
					\dabs{\frac{}{}}{2}
				\end{align*}
			\end{uups}
\end{document}