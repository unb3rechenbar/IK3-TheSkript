\documentclass[../WiSe22ANA3.tex]{subfiles}
\begin{document}
	\lesson{1}{mo 23 jan 2023 10:00}{} 
		\section{Lorentz-Kovariante Formulierung}
			Betrachte nun eine Weltlinie $\gamma\in C^1(\R,\mcM)$ eines Systems $\Sigma_\gamma$. Dann ist die Wegänderung $\diff{\gamma}{h}{t}{}$ unter der Lorentz-Transformation invariant, denn wir können mit Linearität die Ableitung in die Funktionszusammenfassung ziehen:
			$$\diff{\gamma}{h}{t}{}=\fdef{\diff{\gamma_i^*}{h}{t}{}}{i\in[4]}.$$
			Ein Teilchen an $r\in C^1(\R,\mcM)$ mit $\Gamma=r$ als zweites System $\Sigma_\Gamma$ ist dann 
			$$\Maus{L}{s,1}((\gamma-r)(t))=(-v\cdot h,0,0,k(v)\cdot\cmath c_0\cdot h).$$
			\begin{Aufgabe}
				\nr In der Vorlesungsfolie wird $\Maus{L}{s,1}((\gamma-r)(t))=(0,0,0,k(v)\cmath c_0\cdot h)$ behauptet, was ging schief? 
			\end{Aufgabe}
			Daraus können wir die Zeitdilatation ziehen.
			\begin{info}[Zeitdilatation]
				Die Funktionsauswertung $\Funktionenschriftdesign{Relativierung}_v(\cmath c_0t)$ bezeichnen wir als \emph{Zeitdilation}. Sie ist relevant sobald $v\neq 0$. 
			\end{info}
			Für die Längenkontraktion ergibt sich analog
			$$\Funktionenschriftdesign{Relativierung}_v(\gamma_1^*(t_2)-\gamma_1^*(t_1)-v\cdot (t_2-t_1))=:\Funktionenschriftdesign{Relativierung}_v(l)$$
			für $l:=\gamma_1^*(t_2)-\gamma_1^*(t_1)$. 
			\begin{Aufgabe}
				\nr Zeige $v\cdot (t_2-t_1)=0$. 
			\end{Aufgabe}
		
		\section{Beispiele}
			
			
			\subsection{Das Zwillingsparadoxon}	
				\begin{Aufgabe}
					\nr Recherchiere das Paradoxon und mache dir die Problemstellung klar. 
				\end{Aufgabe}
				Die spezielle Relativitätstheorie kann über dieses Paradoxon keine Aussage machen, da dabei mindestens beim Umdrehen kein Inertialsystem mehr vorliegt. 
				
			
\end{document}