\documentclass[../WiSe22ANA3.tex]{subfiles}
\begin{document}
	\lesson{1}{mo 09 jan 2023 10:00}{Gekoppelter Oszillator} 
		\section{Gekoppelter Oszillator}
			Der gekoppelter Oszillator kann durch das Potential $V(x)=kx^2/2$ beschrieben werden. Die GewegungsDGL ist dann $mx''(t)+kx(t)=0_\R$ mit der Lösung $x(t)=\exp(-\lambda t/m)\cdot\Real{A\cdot\exp(-\cmath\omega t)}$.
			
			\begin{Aufgabe}
				\nr Verifiziere die angegebene Lösung $x(t)=\exp(-\lambda t/m)\cdot\Real{A\cdot\exp(-\cmath\omega t)}$ des DGP $mx''(t)+kx(t)=0_\R$ für den gekoppelten Oszillator.
			\end{Aufgabe}
			
		\section{Mehrteilchensysteme}
			\begin{info}[Gleichgewichtszustand]
				Der \emph{Gleichgewichtszustand} ist der Zustand, für welchen alle verallgemeinerten Kräfte gleich $0$ sind, also
				$$0=Q_i(t,(q(t),q'(t)))=:-\Diff{V}{(0,(\mbbEins_i,0)}{t,(q(t),q'(t))}{}.$$
			\end{info}
			Es hat also $V$ einen Extrempunkt $(t,x)\in\kritPkt{V}$. Wir unterscheiden hierbei $(t,q)\in\argmin{V}$ und sagen \enquote{$(t,x)$ ist ein \emph{stabiler Punkt}}, andernfalls mit $(t,q)\in\argmax{V}$ ein \emph{instabiler Punkt}. 
			
			\begin{Aufgabe}
				\nr Bedenke den Zusammenhang des Extrempunktes mit der Definition des Gleichgewichtszustandes. 
			\end{Aufgabe}
			Man kann sich nun die Umgebung des solchen Punktes $(t,q)$ genauer ansehen, indem man kleine Änderungen der Einträge $q_i$ durch eine Verschiebung $h_2(i)$ stimuliert:
			$$\diff{V}{h}{x}{}\approx V(x+h)-V(x).$$
			
			\subsection*{Taylorentwicklung}
			Es ist vielleicht nützlich, sich einmal das ganze aus der \Taylor Perspektive anzusehen, wobei wir die \Einstein Summenkonvention verwenden wollen; Dann ist für $x\in\kritPkt{V}$ 
			$$V(x+h)=V(x)+\Diff{V}{h}{x}{}+\frac{1}{2}\Diff{V}{h}{x}{2}+\TAF{V}{x}{h}.$$
			Dabei ist $\Diff{V}{0,\mbbEins_i}{x}{}=0$ wegen $x\in\kritPkt{V}$ und für $V(x)$ \emph{eichen} wir nach der Art $V(x)=0$, also wählen wir (willkürlich, aber von Vorteil) die lineare Verschiebung $-V(x)$ um schreiben zu können 
			$$V(x+h)=\frac{1}{2}\Diff{V}{h}{x}{2}+\TAF{V}{x}{h}.$$
			Es ist jetzt an der Zeit sich zu überlegen, welche Form $x$ und insbesondere $h$ haben muss: Zunächst ist $x\in\R\times\R^3$, und da wir mit $x$ einen Ort zu einem Zeitpunkt darstellen wollen, wählen wir $q\in C^1(\R,\R^3)$ mit $(t,q(t))=x$ für ein $t\in\R$. Dementsprechend ist die Verschiebung $h$ von der Form $(s,\mch(t))$ für $\mch\in C^1(\R,\R^3)$, also
			$$\tilde q(t)=q(t)+h(t)$$
			für das verschobene $\tilde q$. Wir können schreiben
			$$V((t,q(t))+(s,\mch(t))=\frac{1}{2}\cdot\Diff{V}{(s,\mch(t))}{t,q(t)}{2}+\TAF{V}{(t,q(t))}{s,\mch(t)}.$$
			Dies ist schon um einiges unübersichtlicher, da die Gleichung in die Länge gezogen wird; jedoch bietet es den Vorteil zu überblicken, mit welchen Konstruktionen gearbeitet wird. Wir wollen nun jedoch die Struktur umformulieren. 
			\begin{Aufgabe}
				\nr Führe die Umformulierung durch, indem du die Linearität der Ableitung im Mehrdimensionalen verwendest. Verwende hierbei den \emph{Differentialquotient} $\mbbD_{h}f(x)$.  
			\end{Aufgabe}
			Wir haben dann
			\begin{multline*}
				V(x+(s,\mch(x_1))\approx s^2\cdot\QDiff{V}{\fdef{1_\R,0_{\R^3}}{i\in[2]}}{x}{}+\\
				\sum_{(i,j)\in[3]^2}\stern{\mch}{i}{x_1}\cdot\stern{\mch}{j}{x_1}\cdot\underbrace{\QDiff{V}{((0_\R,e_i),(0_\R,e_j))}{x}{}}_{=d^2V(x)(0_\R,e_i)(0_\R,e_j)}.
			\end{multline*}
			Man fühlt bereits den Raum einer Abkürzung: Wir führen eine Matrix ein, um nicht immer den Differentialquotienten direkt schreiben und sehen zu müssen: 
			$$\Lambda:=\fdef{\QDiff{V}{((0_\R,e_i),(0_\R,e_j))}{x}{}}{(i,j)\in[d]^2}.$$
			Man kann weiter zeigen $\Lambda_{(i,j)}=\Lambda_{(j,i)}$.
			\begin{Aufgabe}
				\nr Mache dir klar, weshalb in den \Taylor Summanden mit $h_i,h_j$ multipliziert wird. Beachte hierbei die Linearität der Ableitung $\Diff{V}{\mbbEins_i}{x}{}$.
				
				\nr Zeige $\Diff{V}{\mbbEins_i}{q}{}=0$ und $V(x)=0$.
				
				\nr Zeige $\Lambda_{(i,j)}=\Lambda_{(j,i)}$. 
			\end{Aufgabe}
			Den kinetischen Energieterm kann man in der Betrachtung des Extrempunktes $(t,q)\in\kritPkt{V}$ zu einer quadratischen Funktion schreiben:
			$$T(x)=\frac{1}{2}\Maus{m}{i,j}\cdot x_ix_j=\frac{1}{2}\Maus{m}{i,j}h_ih_j.$$
			Zu beachten ist, daß $m$ von den Koordinaten $x_k$ abhängt:
			$$\Maus{m}{i,j}(x)=\Maus{m}{q}+...$$
			\begin{Aufgabe}
				\nr Überlege, was $m$ für eine Matrix ist; wie fließt die Auswertung in $x$ in die Konstruktion mit ein?
				
				\nr Vervollständige die Punkte. 
			\end{Aufgabe}
			... Damit können wir dann den Lagrangian schreiben mit
			$$L(t,h(t),h'(t))=\frac{1}{2}\nbra{\Maus{T}{i,j}\stern{h'}{i}{t}\cdot\stern{h'}{j}{t}-\Maus{V}{i,j}\stern{h}{i}{t}\cdot\stern{h}{j}{t}}.$$
			Unter der Verwendung der \Euler-\Lagrange Gleichung erhalten wir die Bewegungsgleichungen, welche Aussagen über die zeitliche Entwicklung der Abweichung $h$ liefern; 
			$$\Diff{L}{0,(0,\mbbEins_l)}{t,x}{}=\frac{1}{2}\Maus{T}{l,i}\cdot\stern{x_2'}{l,i}{j}+\frac{1}{2}\Maus{T}{i,l}\cdot\stern{x_2'}{i}{t}$$
			und wegen $\Maus{T}{i,j}=\Maus{T}{j,i}$ dann 
			$$\Diff{L}{0,(0,\mbbEins_l)}{t,x}{}=\Maus{T}{i,l}\cdot\stern{x_2'}{i}{t}.$$
			Für den Term $\Diff{L}{0,(\mbbEins_i,0)}{(t,x)}{}$ finden wir 
			$$\Diff{L}{0,(\mbbEins_i,0)}{(t,x)}{}=-\Maus{\Lambda}{l,j}\cdot h_j.$$
			Somit ist dann die Bewegungsgleichung 
			$$\Maus{T}{i,j}\stern{h''}{j}{t}+\Maus{\Lambda}{i,j}\stern{h}{j}{t}=0.$$
			\begin{Aufgabe}
				\nr Rechne den Lagrangian nach und stelle ihn formal auf. 
				
				\nr Rechne die Terme der \Euler \Lagrange Gleichung nach und berechne die zeitliche Ableitung. 
			
				\nr Kläre, welche Null in der Bewegungsgleichung gemeint ist.
			\end{Aufgabe}
			... Aus der Theorie der Linearen Algebra sei bekannt, daß das System genau dann eine nicht-triviale Lösung besitzt, wenn
			$$\det(\Lambda-\omega^2 T)=0.$$
			\begin{Aufgabe}
				\nr Zeige die Aussage und den dahinterliegenden Satz. 
			\end{Aufgabe}
			Durch diese Gleichung sind die \emph{Eigenfrequenzen} $\omega$ des Systems. Es kann dabei bis zu $n$ Einträge geben, sogar $\omega\in\C^k\setminus\R^k$ mit $k\in[n]$ ist möglich! Wenn für $i\in[k]$ $\omega_i\in\C$ gilt, dann können wir $\omega_\alpha=\omega'+\cmath \omega''$ mit $\omega',\omega''\in\R$ schreiben. Dann gäbe es (i) für $\omega''>0$ eine \emph{exponentielle Zunahme} oder (ii) für $\omega''<0$ eine \emph{Dämpfung}, was \underline{nicht} mit der Energieerhaltung vereinbar ist. 
			
			\begin{Aufgabe}
				\nr Definiere den Begriff \emph{Eigenfrequenz}. 
				
				\nr Überlege dir, weshalb $\omega''<0$ nicht mit der Energieerhaltung vereinbar ist; Trifft dies auch auf $\omega''>0$ zu?
				
				\nr Was ist $\omega_\alpha$? 
			\end{Aufgabe}
			Damit lässt sich dann die Amplitude $\Maus{A}{i,\alpha}$ bestimmen entlang der Koordinate $\stern{q}{i}{t}$ im \emph{Modus} $\alpha$:
			$$\nbra{\Maus{\Lambda}{i,j}-\omega_\alpha^2\cdot\Maus{T}{i,j}}\cdot\Maus{A}{j,\alpha}=0$$
			mit $\Maus{A}{*,\alpha}=\fdef{\Maus{A}{j,\alpha}}{j\in[n]}$. Wir assoziieren also $\omega_\alpha$ mit der zugehörigen Gleichung der Amplitude $\Maus{A}{j,\alpha}$. 
			\begin{Aufgabe}
				\nr Was ist ein Modus?
			\end{Aufgabe}
			
\end{document}