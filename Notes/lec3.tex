\documentclass[../WiSe22ANA3.tex]{subfiles}
\begin{document}
	\lesson{3}{di 25 okt 2022 10:30}{Wiederholung und Reibung} 
		Wir haben bisher gesehen, daß monogene Systeme, die entweder durch $U(t,(\vec q,\vec q'))$ oder $V(\vec{r})$ beschrieben werden, zu Lagrange-Gleichungen führen. 
		\begin{center}
			\href{https://de.wikibrief.org/wiki/Monogenic_system}{monogen}: auf System wirkende besonders bequem mathematisch modellierbar
		\end{center}
		Der grobe Ablauf wird immer sein
		\begin{clist}
			\item Freiheitsgrade bestimmen mit $\dim(\Raum{\R^n}{\R})\cdot N-k$, wobei $N$ die Teilchenanzahl und $k$ die Anzahl der Zwangsbedingungen/Einschränkungen ist. 
			\item $q$ finden
			\item $L(t,(q,q'))=T(t,(q,q'))-V(t,(q,q'))$ aufstellen
			\item Die partiellen Ableitungen nach den Komponenten bestimmen: $\pdiff{T}{q}{}{}$, $\pdiff{T}{q'}{}{}$, $\pdiff{V}{q}{}{	}$ und $\pdiff{V}{q'}{}{}$.
		\end{clist}
		Dann liefert das Lagrange-PDP mit
		\begin{align*}
			\dv{s}\nbra{\pdv{b}L(s,(a,b))\big|_{\substack{a=q(s)\\b=q'(s)}}}\Bigg|_{s=t}=\pdv{a}L(t,(a,b))\big|_{\substack{a=q(s)\\b=q'(s)}}\numberthis\label{eq:LagragePDP}
		\end{align*}
		die Bewegungsgleichung. Die Gleichung muss $N$ mal gelöst werden.
		\begin{bsp}
			Betrachte ein Partikel in kartesischen Koordinaten mit $V(t,(a,b))=0_\R$	. Setze
			\begin{align*}
				q(t):=x(t)\in\R^3
			\end{align*}
			und $T(t,(q,q'))=m/2\dabs{x'(t)}{2}^2=m/2\dabs{q'(t)}{2}^2$. Damit folgt dann für $i\in[3]$ 
			\begin{align*}
				&\pdv{a_i}T(t,(a,q'(t)))|_{a=q(t)}=0_\R, &&\pdv{b_i}T(t,(q(t),b))|_{b=q'(t)}=mq_i'(t).
			\end{align*}
			Damit ist in \eqref{eq:LagragePDP} die rechte Seite des Systems immer $0_\R$, sodaß
			\begin{align*}
				\dv{s}mq_i'(s)|_{s=t}=0_\R=:mq''_i(t).
			\end{align*}
			\underline{\textbf{Zwangsbedingungen.}} Setze nun $x_2(t),x_3(t)=0_\R=q_2(t),q_3(t)$ voraus. Mit einem Potential 
			\begin{align*}
				V(t,(q(t),q'(t)))=\scpr{\fdef{F_i(q_i(t))}{i\in[3]}}{q(t)},\qquad F_i:\R\to\R
			\end{align*}
			gilt dann $V(t,(q(t),q'(t))=F_1(q_1(t))\cdot q_1(t)$. Setze ein affinlineares Potential voraus, dann gilt insgesamt
			\begin{align*}
				V(t,(q(t),q'(t)))=F_1(q_1(t))\cdot q_1(t)+V_0.
			\end{align*}
			Damit ist dann die rechte Seite in \eqref{eq:LagragePDP} nicht mehr gleich $0_\R$ und wir erhalten
			\begin{multline*}
				\pdv{a_1}V(t,(a,q'(t)))\big|_{a=q(t)}\\
				=\nbra{a_1\cdot \pdv{a_1}F_1(a_1)+F_1(a_1)\cdot\pdv{a_1}a_1}\bigg|_{a=q(t)} \\
				=\nbra{F_1'(a_1)\cdot a_1+F_1(a_1)}\big|_{a=q(t)} \\
				=F_1'(q_1(t))\cdot q_1(t)+F_1(q_1(t)).
			\end{multline*}
			Damit ergibt sich mit \eqref{eq:LagragePDP} gerade
			\begin{align*}
				& mq_1''(t)=F_1'(q_1(t))\cdot q_1(t)+F_1(q_1(t)) &&mq_i''(t)=0_\R
			\end{align*}
			für $i\in\nset{2,3}$. Von hieraus könnte man den Exponentialansatz zur Lösung verwenden. 
 		\end{bsp}
 		\begin{bsp}
 			Betrachte zwei Massen im Gravitationsfeld, welche mit einem Seil über eine fixe Rolle miteinander verbunden sind. In drei Dimensionen wären $3\cdot 2$ Freiheitsgrade möglich, jedoch bewegen sich die Objekte nur in einer Linie, also $1\cdot 2=2$. Hat das Seil die Länge $l$ und die Massen die Abstände $x_1(t),x_2(t)$ zur Rotationsachse der Rolle, so gilt 
 			\begin{align*}
 				x_1(t)+x_2(t)=l\Longleftrightarrow x_2(t)=l-x_1(t). 
 			\end{align*}
 			Damit sind die Ableitungen bis auf Vorzeichen gleich: $x_1'(t)=-x_2'(t)$. Insbesondere kann $x_2(t)$ durch $x_1(t)$ ausgedrückt werden, wodurch nur noch ein Freiheitsgrad besteht! Setze dann $q(t):=x_1(t)$ und schreibe
 			\begin{align*}
 				T(t,(q(t),q'(t)))=\frac{1}{2}q'(t)^2\cdot\sum_{i\in[2]}m_i.
 			\end{align*}
 			\underline{\textbf{Potentialsuche.}} Suche ein passendes Potential $V$, welches das System beschreibt. Wähle die Summe der potentiellen Energien der einzelnen Teilchen, also
 			\begin{multline*}
 				V(t,(q(t),q'(t)))=-m_1g\cdot q(t)-m_2g\cdot (l-q(t)) \\
 				=q(t)\cdot g\cdot (m_2-m_1)-m_2gl.
 			\end{multline*}
 			Dann folgt die Lagrangegleichung mit
 			\begin{align*}
 				\frac{1}{2}q'(t)^2\cdot(m_1+m_2)=q(t)g\cdot(m_2-m_1)-m_2gl
 			\end{align*}
 			und mit \eqref{eq:LagragePDP} dann schließlich
 			\begin{align*}
 				q''(t)\cdot(m_1+m_2)=g\cdot(m_2-m_1).
 			\end{align*}
  		\end{bsp}
  	
  	\section{Reibung}
		DGPs der Form
		\begin{align*}
			mx''(t)-\gamma x'(t)+kx(t)=0
		\end{align*}
		lassen sich für gewöhnlich mit dem Exponentialansatz 
		\begin{align*}
			x(t)=\exp(\cmath\sqrt{\frac{k}{m}t}-\frac{\gamma}{2}t) 
		\end{align*}
		lösen. Als eine weitere wichtige, \so{nicht konservative} Kraft ist die Reibung zu betrachten. Für ein holonomes System lassen sich die Lagrange-Gleichungen immer in die Form
		\begin{align*}
			\dv{s}\nbra{\pdv{b}L(s,(a,b))\big|_{\substack{a=q(s)\\b=q'(s)}}}\Bigg|_{s=t}-\pdv{a}L(t,(a,b))\big|_{\substack{a=q(s)\\b=q'(s)}}=Q\qquad Q\in\R
		\end{align*}
		bringen. 
		\begin{bsp}
			Reibung entlang der $e_1$-Achse. Bewegt sich ein Teilchen auf dieser und erfährt Reibung, die proportional zu seiner Geschwindigkeit $x_1'(t)$ ist, dann können wir schreiben
			\begin{align*}
				F_1(t)=-k_1 x_1'(t)=-\pdv{a}\nbra{\frac{1}{2}k_1 a^2}\bigg|_{a=x_1'(t)}
			\end{align*}	
			und für den gesamten Vektor
			\begin{align*}
				F(t)=\fdef{-k_i x_i'(t)}{i\in[3]} \qquad k_i\in\R.
			\end{align*}
		\end{bsp}
		Diesen können wir über die \textsc{Rayleigh}sche Dissipationsfunktion ausdrücken:
		\begin{align*}
			F(t)=-\div f(b)|_{b=x'(t)}:=-\fdef{\pdv{b_i}\nbra{\frac{k_i}{2}\cdot b_i^2}\bigg|_{b_i=x_i'(t)}}{i\in[3]}.\numberthis\label{eq:ReibungRayleigh} 
		\end{align*}
		Die genannte Funktion ist für $N$ Teilchen allgemeiner
		\begin{align*}
			f(v(t)):=\frac{1}{2}\sum_{i\in[N]}\nbra{k_{(i,1)}v_{(i,1)}(t)^2+k_{(i,2)}v_{(i,2)}(t)^2+k_{(i,3)}v_{(i,3)}(t)^2},
		\end{align*}
		wobei $v(t):=\fdef{x_i'(t)}{i\in[N]}$ und $x_i(t)\in\R^3$, sowie $k\in\R^{N\times 3}$. Man kann nun die generalisierte Kraft $Q_j$ nach dessen Definition so ausdrücken, daß die Reibung berücksichtigt wird:
		\begin{multline*}
			Q_j:=\sum_{i\in[N]}\scpr{\vec F_i}{\pdiff{\vec r_i}{q_j}{t,(q(t),q'(t))}{}}\\
			\stackrel{\eqref{eq:ReibungRayleigh}}{=}-\sum_{i\in[N]}\scpr{\div f(b)|_{b=x'(t)}}{\pdiff{\vec r_i}{q_j}{t,(q(t),q'(t))}{}}
			\\\stackrel{(*)}{=}-\sum_{i\in[N]}\scpr{\div f(b)|_{b=x'(t)}}{\pdiff{\vec r'_i}{q'_j}{t,(q(t),q'(t))}{}} 
			\\=-\pdiff{f}{q'_j}{q'(t)}{} 
		\end{multline*}
		mit (*) $\\pdiff{\vec r_i}{q_j}{t,(q(t),q'(t))}{}=\pdiff{\vec r'_i}{q'_j}{t,(q(t),q'(t))}{}$. Damit folgt dann die Lagrangegleichung für Reibung.
		\begin{info}[Lagrange-DGL mit Reibung]
			Bezieht man die Reibungskraft in die Bewegung mit ein, folgt 
			\begin{align*}
				\dv{s}\nbra{\pdv{b}L(s,(a,b))\big|_{\substack{a=q(s)\\b=q'(s)}}}\Bigg|_{s=t}-\pdv{a}L(t,(a,b))\big|_{\substack{a=q(s)\\b=q'(s)}}=-\pdiff{f}{q'_j}{q'(t)}{}.
			\end{align*}
		\end{info}
\end{document}