\documentclass[../WiSe22ANA3.tex]{subfiles}
\begin{document}
	\lesson{1}{mo 30 jan 2023 10:00}{}
	\section{Maxwellsche Gleichung}
	
		\begin{Beispiel}[Gegenläufige Objektkollision]
			Seien $p_1,p_2:\R\to\mcM$ mit $p_1=-p_2$. Dann ist im Minkowskiraum die Summe
			$$p_1+_\mcM p_2=\fdef{\begin{cases}
				(p_1)^*_i-(p_2)^*_i & i\in [3] \\
				\cmath/c_0\cdot(E_1+E_2) \sonst
			\end{cases}}{i\in[4]}.$$
		\end{Beispiel}
		Aus dem Beispiel folgern wir für die Norm des resultierenden Impulses $\tilde p\in\mcM$ zu einem Zeitpunkt $t\in\R$
		$$\dabs{\tilde p(t)}{2}:=\dabs{\fdef{\tilde p_i^*(t)}{i\in[3]}}{2}+\nbra{\cmath/c_0\cdot(E_1(t)+E_2(t))}^2.$$
		
		
	\section{Relativistische Mechanik und Kinematik}
		\begin{info}[Die Schwellenenergie]
			
		\end{info}
\end{document}