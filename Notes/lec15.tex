\documentclass[../WiSe22ANA3.tex]{subfiles}
\begin{document}
	\lesson{1}{mo 16 jan 2023 10:00}{Spezielle Relativitätstheorie Einleitung} 
		\textbf{Ziele des Kapitels:}
		\begin{clist}
			\item Konzepte der \Lorentz-Transformationen, der \Lorentz-Invarianz und des \Minkowski-Raumes verstehen.
			\item Mit $4$-Vektoren und ihren mathematischen Eigenschaften vertraut sein.
			\item In der Lage sein, elementare Probleme der relativistischen Mechanik zu lösen, wie z.B. die Streuung von Teilchen.
			\item Den Begriff der Schwellenenergie erklären können und wissen, wie sich elektromagnetische Felder relativistisch transformieren. 
		\end{clist}
		
		\section{Einleitung und Postulate}
			Wir betrachten ein durch die Länge $L\in\R$ charakterisiertes System in $\R^3$. Ein in diesem befindlicher Körper der Masse $m\in\R$ habe eine Geschwindigkeit $v\in\R^3$ und damit den Impuls $p\in\R^3$. 
			\begin{info}[De-Broglie Wellenlänge]
				Die de-Broglie Wellenlänge ordnet jedem Körper mit dem Impuls $p\in\Abb{\R}{\R^3\setminus\nset{\Null{\R^3}}}$ eine charakteristische Wellenlänge $\lambda\in\R$ zu:
				$$\lambda(t):=\frac{h}{p(t)}$$
			\end{info}
			\begin{Aufgabe}
				\nr Was ist $L\in\R$ für eine charakterisierende Zuordnungsgröße?
			\end{Aufgabe}
			Für alle weiteren Überlegungen definieren wir zunächst das Newtonsche Inertialsystem, welches eine Basis von $\R^3$ mit charakterisierenden Forderungen an den Körper ist. 
			\begin{info}[Newtonsches Inertialsystem]
				Als \emph{Newtonsches Intertialsystem} bezeichnen wir ein Koordinatensystem $\underline v\in\Basis{\R^d}{\R}$, in welchem sich genau dann alle Objekte in geradliniger unbeschleunigter Bewegung befinden, wenn $\vec F=0$. 
			\end{info}
			\begin{Aufgabe}
				\nr Überlege dir Syteme, welche die Intertialbedingung nicht erfüllen. Betrachte dazu die Coriolis-Kraft und Koordinatentransformationen. 
			\end{Aufgabe}
			\Einstein formulierte also die folgenden beiden Postulate als axiomatische Begründung der speziellen Relativitätstheorie:
			\begin{clist}
				\item Die Gesetze der Physik sind in allen Inertialsystemen gleich. 
				\item Die Lichtgeschwindigkeit $c_0$ ist bewegungsunabhängig und hat daher in allen Inertialsystemen denselben Wert: $c_0=299792458\si{\metre\per\second}$. 
			\end{clist}
		
		\section{Minkowski-Raum}
			Wir verwenden für die folgenden Definitionen die in der Vorlesung eingeführte Konvention mit komplexer Zeitdarstellung $\cmath\cdot c_0\cdot t\in\C$. Auf andere Konventionen kommen wir kurz zu sprechen.  
			\begin{info}[Minkowskivektormenge]
				Als \emph{Minkowski-Raum}, manchmal auch \emph{Weltraum} genannt, bezeichnen wir den Vektorraum auf der Vektorenmenge
				$$\mcM:=\nset{x\in\C^4:x_1,x_2,x_3\in\R}$$
				mit $\R$ als Multiplikatormenge. Einen Weg auf $\mcM$ nennen wir \emph{Weltlinie}. 
			\end{info}
			\begin{Aufgabe}
				\nr Zeige $\mcM$ bildet mit $K$ Körper einen Vektorraum. 
			\end{Aufgabe}
			\noindent Wir führen weiter den Raumzeitvektor ein, ein Element aus $\mcM$, welcher für die Relativitätstheorie eine grundlegende Rolle spielen und zur Formulierung der Konzepte ohne Betrachtung der Gravitation hilfreich sein wird.
			\begin{info}[Der Raumzeitvektor]
				Habe ein Objekt die Ortszuweisung $r:\R\to\R^3$, also den Ort $r(t)$ zum Zeitpunkt $t\in\R$, dann nennen wir
				$$\Mengenschriftdesign{Raumzeitvektor}_{r}(t,r(t)):=(\stern{r}{1}{t},\stern{r}{2}{t},\stern{r}{3}{t},\cmath c_0\cdot t)\in\mcM$$
				den Raumzeitvektor des Objektes bezüglich $r$.  
			\end{info}
			\noindent Hier können wir gleich eine Charakterisierung vorziehen, die eigentlich erst in \texttt{lec17} stattfinden würde; Nämlich müssen wir uns Gedanken über die im Minkowski-Raum verwendete Metrik machen. 
			\subsection*{Metrikkonstruktion}
				Diese wird erzeugt durch eine Funktion $g$ der Form $g:A\to\Abb{V^2}{\R}$. Zunächst betrachten wir die Menge $A$: Sie ist Teil des Tupels $(A,V,\rightarrow)$, genannt \emph{affiner Punktraum}. 
				\begin{info}[Affiner Punktraum und metrischer Tensor]
					Sei $A$ eine nichtleere Menge, $V$ ein $K-$Vektorraum und $\rightarrow:A^2\to V$. Hat die Abbildung $\rightarrow$ die Eigenschaften (i) $\forall P,Q,R\in A:\ovec{PQ}+\ovec{QR}=\ovec{PR}$ und (ii) $\forall v\in V,P\in A:\exists_1 Q\in A:\ovec{PQ}=v$, dann nennen wir das Tipel $(A,V,\rightarrow)$ einen \emph{affinen Punktraum} und schreiben $(A,V,\rightarrow)\in\affinPR{A}{V}{K}$. 
				\end{info} 
				\noindent An dieser macht es Sinn, die Tensordefinition zu betrachten.
				\begin{info}[Tensor und Tensorräume]
					Sei $K$ ein Körper und $\fdef{V_i}{i\in I}$ Vektormengen. Sei $\Raum{V_i}{K}$ endlichdimensional und $N:=\card(I)\in\N$. Dann nennen wir für $\mbbV:=\prod_{i\in I}V_i$ die Funktion $\Phi:\mbbV\to K$ einen \emph{Tensor}, wenn $\Phi\in\MultiLinearform{\mbbV}{K}$. 
				\end{info}
				\noindent Für den Spezialfall $V_i=V_j$ für $(i,j)\in I^2$ definieren wir 
				$$\bigotimes^NV^*:=\MultiLinearform{V^N}{K}$$
				für $N:=\card(I)\in\N$ mit $\dim(\Raum{\bigotimes^NV^*}{K})=\dim(V)^N$. Weiter gilt sogar $\bigotimes^NV^*\cong\R^{\dim(V)^N}$. 
				\begin{Aufgabe}
					\nr Definiere die Mengen $\MultiLinearform{\mbbV}{K}$ nach dem Bauplan der aus LinA 2 bekannten Menge $\Bilinearform{\mbbV}{K}$ im Fall $\mbbV=\prod_{i\in[2]}V_i$.
					
					\nr Rechne die Vektorraumaxiome für $\bigotimes^NV^*$ als $K$-Vektorraum nach. 
					
					\nr Zeige $dim(\Raum{\bigotimes^NV^*}{K})=\dim(V)^N$ und damit $\bigotimes^NV^*\cong\R^{\dim(V)^N}$. 
					
					\nr Stelle das Tensorprodukt $\Psi\otimes\Phi$ für $\Psi,\Phi\in\MultiLinearform{\mbbV}{K}$ auf. Zeige die Distributivität, Assoziativität und Kommutativität. 
				\end{Aufgabe}
				\noindent Damit können wir eine \emph{nicht-ausgeartete, symmetrische Bilinearform} zur Multiplikation zweier Minkowski-Vektoren definieren: Für jedes $P\in A$ und $\Raum{V}{K}:=\Raum{\mcM}{\R}$ ist 
				$$\Maus{\mal}{\mcM,P}(x,y):=g(P)(x,y)$$
				mit $x,y\in\mcM$. Wegen der im Allgemeinen fehlenden positiv Definitheit ist jedoch $g(P)$ kein Skalarprodukt auf $\Raum{\mcM}{\R}$! Um dennoch über Abstände und Winkel sprechen zu können, nutzt man in abgeschwächter Metrikbedingung (keine pd) den \emph{pseudometrischen Tensor} der Art
				$$\Maus{\mal}{\mcM,P}(x,y)=x^T\circ\eta\circ y.$$
				\begin{Aufgabe}
					\nr Kläre, welcher affiner Punktraum $(A,V,\rightarrow)$ für den Spezialfall $V:=\mcM$ und $K:=\R$ gewählt wird. Wie sieht die Abbildung $\rightarrow$ aus?
				
					\nr Zeige $g(P)$ ist (i) nicht ausgeartet, (ii) symmetrisch und (iii) bilinear, insbesondere $g(P)\in\Bil{\mcM}$. Verwende dabei $\nset{v\in V:\forall w\in V:b(v,w)=0}=\nset{0_V}$.
				\end{Aufgabe}
				\noindent Für die Bilinearform $g(P)$ finden wir nun eine (konventionsabhängige) eindeutige Darstellungsmatrix 
				$$G(P)=\fdef{g(P)(e_i,e_j)}{(i,j)\in[4]}=(1,1,1,1)\cdot I_4$$
				mit $I_4$ als Einheitsmatrix. Sie hat die $\Spur{G(P)}=4$ und $(4,0)$ als Sylvester-Signatur. Alternativ für $(\stern{\gamma}{1}{t},\stern{\gamma}{2}{t},\stern{\gamma}{3}{t},c_0t)$ als Raumzeitvektor mit angepasstem $\mcM$ dann $G(P)=(1,1,1,-1)\cdot I_4$ mit $(3,1)$ als Signatur. 
				\begin{Aufgabe}
					\nr Berechne die Spur der Matrix $G(P)$ und bestimme ihre Sylvester-Signatur. Schreibe die Matrix einmal aus. 
					
					\nr Zeige, daß alle konventionellen Darstellungsarten äquivalent sind. 
				\end{Aufgabe}
				\noindent Mit Blick auf den \Mengenschriftdesign{Raumzeitvektor} klassifizieren wir folgendermaßen.  
				\begin{info}[Raumvektorklassen]
					Sei $x\in\mcM$. Wir bezeichnen $\dabs{x}{P}^2\in\R_{>0}$ als \emph{raumähnlich}, $\dabs{x}{P}^2\in\R_{<0}$ als \emph{zeitlich} und $\dabs{x}{P}^2=0_\R$ als \emph{lichtartig}. 
				\end{info}
			
			\subsection*{Geschwindigkeiten}
				Mit dem konstruierten Raumzeitvektor des Ortes eines Objektes können wir den zugrundelegenden Weg ableiten und erhalten die Geschwindigkeit. Wir verketten die noch mit der Relativierungsfunktion:
				\begin{info}[Relativierung]
					Wir defineren die \emph{Relativierungsfunktion} als die Abbildung $f:V\to W$ auf den $K$-Vektorräumen $V,W$ der Form 
					$$x\mapsto \mal_{W}(k(v),x)$$
					für $k(v):=1/\sqrt{1-\dabs{v}{2}^2/c_0^2}\in\R$ als \emph{Relativierungskoeffizient}, wobei mit $v$ meist die Geschwindigkeit $||\fdef{\stern{\gamma'}{i}{t}}{i\in[3]}||_2$ für einen Weg $\gamma:\R\to\R^3$ gemeint ist. 
				\end{info}
				\noindent Dann können wir schreiben 
				$$v(t):=(\Funktionenschriftdesign{Relativierung}\,\circ D^1\circ \Funktionenschriftdesign{Raumzeitvektor}_r)(t,r(t))$$
				für $r:\R\to\R^3$ und $t\in\R$. Mit $D^1$ meinen wir die gewöhnliche formale Ableitung. In der Physik benutzt man häufig die Abkürzung $d\gamma/d\tau$, wobei $\gamma$ eine Weltlinie in $\mcM$ ist. 
				\begin{Aufgabe}
					\nr Wende das Konzept auf den konkreten Ort $r(t):=(x(t),y(t),z(t))$ an und berechne $v(t)$. 
					
					\nr Zeige nun $\Maus{\scpr{x}{x}}{\mcM,P}<0_\R$ mit $\dabs{\fdef{x_i}{i\in[3]}}{2}<c_0$. 
				\end{Aufgabe}
\end{document}